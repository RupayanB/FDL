\documentclass[11pt]{article}

\usepackage[gray]{xcolor}
\usepackage{minted}
\usepackage[USenglish]{babel}
\usepackage[margin=0.7in]{geometry}
\usepackage{inconsolata}
\usepackage{changepage}
\usepackage{array}

% Title Definition
\title{\textbf{\huge Proposal: File Management Language (FML)}}
\author{
  Cara J. Borenstein\\
  Columbia University\\
  cjb2182@columbia.edu
  \and
  Daniel Garzon\\
  Columbia University\\
  dg2796@columbia.edu
  \and
  Daniel L. Newman\\
  Columbia University\\
  dln2111@columbia.edu
  \and
  Pranav Bhalla\\
  Columbia University\\
  pb2538@columbia.edu
  \and
  Rupayan Basu\\
  Columbia University\\
  rb3034@columbia.edu
}

\begin{document}

\maketitle

\begin{abstract}
This is the paper's abstract \ldots
\end{abstract}

\section{Introduction}
This is an \textbf{introduction} to our \emph{language} \ldots

\paragraph{Language Goals}
The remainder of this article is organized as follows.
Section~\ref{previous work} gives account of previous work.
Our new and exciting results are described in Section~\ref{results}.
Finally, Section~\ref{conclusions} gives the conclusions.

\newpage
\section{Syntax}\label{previous work}

\subsection{Basic Data Types}

\begin{table}[ht]
  \centering
  \caption {Basic Data Types} \label{tab:title}
  \vspace{0.5em}
  \begin{tabular}{|m{8em}|b{25em}|}
  \hline

  %% row 1
  \textbf{primitive}
  &
  \vspace{0.7em}
  \textbf{Description} \\ [0.7em]
  \hline

  %% row 2
  \emph{integer}
  &
  \vspace{0.7em}
  \dots Some text
  \\[0.7em]
  \hline

  %% row 3
  \emph{boolean}
  &
  \vspace{0.7em}
  \dots
  \\[0.7em]
  \hline

  %% row 4
  \emph{character}
  &
  \vspace{0.7em}
  \dots
  \\[0.7em]
  \hline

  %% row 5
  \emph{string}
  &
  \vspace{0.7em}
  \dots
  \\[0.7em]
  \hline

  %% row 6
  \emph{directory}
  &
  \vspace{0.7em}
  \dots
  \\[0.7em]
  \hline

  %% row 7
  \emph{file}
  &
  \vspace{0.7em}
  \dots
  \\[0.7em]
  \hline

  %% row 8
  \emph{date}
  &
  \vspace{0.7em}
  \dots
  \\[0.7em]
  \hline

  %% row 9
  \emph{ftype}
  &
  \vspace{0.7em}
  \dots
  \\[0.7em]
  \hline

  %% row 10
  \emph{tag}
  &
  \vspace{0.7em}
  \dots
  \\[0.7em]
  \hline
  \end{tabular}
\end{table}

\subsection{Mathematical Operators}

\begin{table}[ht]
  \centering
  \caption {Mathematical Operators.} \label{tab:title}
  \vspace{0.5em}
  \begin{tabular}{|m{8em}|b{25em}|}
  \hline

  %% row 1
  \textbf{operator}
  &
  \vspace{0.7em}
  \textbf{Description} \\ [0.7em]
  \hline

  %% row 2
  $+$
  &
  \vspace{0.7em}
  \dots Some text
  \\[0.7em]
  \hline

  %% row 3
  $-$
  &
  \vspace{0.7em}
  \dots
  \\[0.7em]
  \hline

  %% row 4
  $*$
  &
  \vspace{0.7em}
  \dots
  \\[0.7em]
  \hline

  %% row 5
  $,$
  &
  \vspace{0.7em}
  \dots
  \\[0.7em]
  \hline

  \end{tabular}
\end{table}

\newpage

\subsection{Logical Operators}

\begin{table}[ht]
  \centering
  \caption {Logical Operators.} \label{tab:title}
  \vspace{0.5em}
  \begin{tabular}{|m{8em}|b{25em}|}
  \hline

  %% row 1
  \textbf{operator}
  &
  \vspace{0.7em}
  \textbf{Description} \\ [0.7em]
  \hline

  %% row 2
  $==$
  &
  \vspace{0.7em}
  \dots Some text
  \\[0.7em]
  \hline

  %% row 3
  $!=$
  &
  \vspace{0.7em}
  \dots
  \\[0.7em]
  \hline

  %% row 4
  $>$
  &
  \vspace{0.7em}
  \dots
  \\[0.7em]
  \hline

  %% row 5
  $>=$
  &
  \vspace{0.7em}
  \dots
  \\[0.7em]
  \hline

  %% row 6
  $<$
  &
  \vspace{0.7em}
  \dots
  \\[0.7em]
  \hline

  %% row 7
  $<=$
  &
  \vspace{0.7em}
  \dots
  \\[0.7em]
  \hline

  \end{tabular}
\end{table}

\subsection{Control Statements}
\subsubsection{\emph{if-then-else}}
\begin{listing}[H]
  \definecolor{bg}{rgb}{0.95,0.95,0.95}
  \begin{minted}[linenos=false, bgcolor=bg, tabsize=2, fontsize=\normalsize,mathescape]{python}
    if <condition> then
      <expression>
    else
      <expression>
    end
  \end{minted}
  \label{lst:the-code}
\end{listing}

\subsubsection{\emph{while}}
\begin{listing}[H]
  \definecolor{bg}{rgb}{0.95,0.95,0.95}
  \begin{minted}[linenos=false, bgcolor=bg, tabsize=2, fontsize=\normalsize,mathescape]{c}
    while <condition> then
      <expression>
    end
  \end{minted}
  \label{lst:the-code}
\end{listing}

\subsubsection{\emph{for}}
\begin{listing}[H]
  \definecolor{bg}{rgb}{0.95,0.95,0.95}
  \begin{minted}[linenos=false, bgcolor=bg, tabsize=2, fontsize=\normalsize,mathescape]{c}
    for <identifier> in <set of files> do
      <expression>
    end
  \end{minted}
  \label{lst:the-code}
\end{listing}
\subsection{Function Definition}
\begin{listing}[H]
  \definecolor{bg}{rgb}{0.95,0.95,0.95}
  \begin{minted}[linenos=false, bgcolor=bg, tabsize=2, fontsize=\normalsize,mathescape]{c}
    def <identifier> (<parameter list>)
      <expression>
    end
  \end{minted}
  \label{lst:the-code}
\end{listing}

\subsection{Built-in Functions}

\section{Example Program}\label{results}
In this section we describe the results.
\begin{listing}[H]
  \caption{Code for ...}
  \definecolor{bg}{rgb}{0.95,0.95,0.95}
  \begin{minted}[linenos=true, bgcolor=bg, tabsize=2, fontsize=\normalsize,mathescape]{c}
    int main(int argc, char const *argv[])
    {
      printf("%s\n", "Hello World!");
      return 0;
    }
  \end{minted}
  \label{lst:the-code}
\end{listing}

\section{Conclusions}\label{conclusions}
We worked hard, and achieved very little.

\end{document}