% Title Definition
\title{\textbf{\huge Reference Manual: ...}}
\author{
  Cara J. Borenstein\\
  Columbia University\\
  cjb2182@columbia.edu
  \and
  Daniel Garzon\\
  Columbia University\\
  dg2796@columbia.edu
  \and
  Daniel L. Newman\\
  Columbia University\\
  dln2111@columbia.edu
  \and
  Pranav Bhalla\\
  Columbia University\\
  pb2538@columbia.edu
  \and
  Rupayan Basu\\
  Columbia University\\
  rb3034@columbia.edu
}
\date{}

\documentclass[10pt]{article}

\usepackage[gray]{xcolor}
\usepackage{minted}
\usepackage[USenglish]{babel}
\usepackage[margin=0.7in]{geometry}
\usepackage{inconsolata}

\begin{document}

\maketitle

\begin{abstract}
This is the paper's abstract \ldots
\end{abstract}

\section{Introduction}
This is an \textbf{introduction} to our \emph{language} \ldots

\paragraph{Language Goals}
The remainder of this article is organized as follows.
Section~\ref{previous work} gives account of previous work.
Our new and exciting results are described in Section~\ref{results}.
Finally, Section~\ref{conclusions} gives the conclusions.

\section{Syntax}\label{previous work}

\subsection{Basic Data Types}
\subsection{Mathematical Operators}
\subsection{Logical Operators}
\subsection{Valid Expressions}
\subsection{Control Statements}
\subsection{Function Definition}

\section{Example Program}\label{results}
In this section we describe the results.
\begin{listing}[H]
  \caption{Code for ...}
  \definecolor{bg}{rgb}{0.95,0.95,0.95}
  \begin{minted}[linenos=true, bgcolor=bg, tabsize=2, fontsize=\normalsize,mathescape]{c}
    int main(int argc, char const *argv[])
    {
      printf("%s\n", "Hello World!");
      return 0;
    }
  \end{minted}
  \label{lst:the-code}
\end{listing}

\section{Conclusions}\label{conclusions}
We worked hard, and achieved very little.

\bibliographystyle{abbrv}
\bibliography{main}

\end{document}